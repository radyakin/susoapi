\section{Assignments}
\putimage{pexels-pixabay-164686.jpg}





\subsection{Get a list of assignments on the server}
\begin{lstlisting}[style=CommandLineStyle, showlines=true]

.get_assignments ,
  [qguid(string) qversion(integer)]
  [responsible(string) supervisorid(string)]
  [showarchive frame(string)]

\end{lstlisting}

\optsheader
\begin{itemize}

  \item \option{qguid} - questionnaire GUID identifier (optional, if not
  specified, no filtering by questionnaire or version is done).

  \item \option{qversion} - questionnaire version (optional, if not specified 1
  is assumed, only effective when questionnaire GUID is specified).

  \item \option{responsible} - optional responsible name or GUID.

  \item \option{supervisorid} - optional supervisor GUID (not username).

  \item \option{showarchive} - regulates whether to search among active
  assignments (when this option is not specified, default) or archived
  assignments (when this option is specified).

  \item \option{frame} - (optional) name of the frame to save the list of the
  interviews.

\end{itemize}

\structheader{Assignments list}
\begin{compactitem}

  \datafield{long id } - for example:
  \xmpl{1064}

  \datafield{str36 responsible\_id } - identifier (GUID) of a person designated
  responsible for the assignment, for example:
  \xmpl{"1e7ef1e1-2123-4b5f-8a34-72ad1c494d1b"}

  \datafield{strL responsible\_name } - account name of a person designated
  responsible for the assignment, for example:
  \xmpl{"Natalia"}

  \datafield{strL questionnaireid } - for example:
  \xmpl{"8f1286c40c90475a9100b618584410b6\$1"}

  \datafield{long interviews\_count } - number of interviews assignment calls
  for, for example:
  \xmpl{10}

  \datafield{long quantity } - for example:
  \xmpl{1}

  \datafield{byte archived } - 1 if the assignment is archived, 0 if not
  archived (active), for example:
  \xmpl{0}

  \datafield{strL created\_at\_utc } - timestamp for when the assignment was
  created, in UTC, for example:
  \xmpl{"2024-03-19T23:48:04.358641Z"}

  \datafield{strL updated\_at\_utc } - timestamp for when the assignment was
  last updated, in UTC, for example:
  \xmpl{"2024-03-19T23:48:04.358641Z"}

  \datafield{strL email } - (for web assignments only) the email of the
  assignment recipient (respondent), for example:
  \xmpl{"somebody@somewhere.org"}

  \datafield{strL password } - (for web assignments only) the password to open
  the assignment (begin interview), for example:
  \xmpl{"PASS1234"}

  \datafield{byte webmode } - flag which indicates if the assignment is a web
  assignment (value is 1), or not (value is 0), for example:
  \xmpl{1}

  \datafield{strL received\_by\_tablet\_at\_utc } - timestamp for when the
  assignment was received by a mobile device, in UTC, for example:
  \xmpl{"2024-03-19T23:48:04.358641Z"}

  \datafield{byte is\_audio\_recording\_enabled } - 1 if audio audit is enabled
  for the assignment, 0 if not enabled, for example:
  \xmpl{0}

\end{compactitem}


\subsection{Get details of an assignment}

\begin{lstlisting}[style=CommandLineStyle, showlines=true]

.assignments_getdetails assignmentid

\end{lstlisting}

\paramsheader
\begin{itemize}
    \item \option{assignmentid} - numeric identifier of the assignment, for
    example:
    \xmpl{1007}
\end{itemize}

\savedres
\begin{compactitem}

    \item \savedresult{r(CreatedAtUtc)} - timestamp (date and time) of the
    creation of this assignment (in UTC), for example:
    \xmpl{"2021-05-17T14:41:46.824627Z"}

    \item \savedresult{r(UpdatedAtUtc)} - timestamp (date and time) of the last
    update (in UTC), for example:
    \xmpl{"2021-05-27T20:45:39.778723Z"}

    \item \savedresult{r(QuestionnaireId)}  - identity of the questionnaire on
    which this assignment is based, for example:
    \xmpl{"0434573c67b34d93b1f0799fb042f9e6\$1"}

    \item \savedresult{r(ResponsibleName)} - login name of the current
    responsible user, for example:
    \xmpl{"Natalia"}

    \item \savedresult{r(ResponsibleId)} - GUID of the current responsible user,
    for example:
    \xmpl{"71dc5a0f-6557-4945-9b86-0197699ee475"}

\end{compactitem}

\vskip16pt
Additionally, for each identifying question, \#=1,2,...:
\begin{compactitem}

    \item \savedresult{r(id\_guid\#)} - internally assigned GUID of the question,
    for example:
    \xmpl{"c03979a9157304fcfa9bc1d3988433d7"}

    \item \savedresult{r(id\_variable\#)} - variable name for the question,
    for example:
    \xmpl{"province"}

    \item \savedresult{r(id\_answer\#)} - recorded answer to the identifying
    question, for example:
    \xmpl{"Northern province"}

\end{compactitem}

\errheader
\begin{itemize}
    \item Error \ecode{5001} is returned if the \textit{assignmentid} is not specified.
    \item Error \ecode{5101} is returned if the \textit{assignmentid} is not a number (expected to be numeric).
    \item Error \ecode{5102} is returned if the \textit{assignmentid} is not 1 or more (expected to be 1 or a larger value).
    \item Error \ecode{5103} is returned if the \textit{assignmentid} is not an integer (expected to be an integer).
    \item Error \ecode{-10404} is returned if an assignment with \textit{assignmentid} is not found on the server.
\end{itemize}


\subsection{Assign an assignment to a new responsible}

\begin{lstlisting}[style=CommandLineStyle]
.assignments_assign assignmentid responsiblelogin
\end{lstlisting}

\paramsheader
\begin{itemize}

    \item \option{assignmentid} - numeric identifier of the assignment, for
    example:
    \xmpl{1007}

    \item \option{responsiblelogin} - account name of the new responsible, for
    example:
    \xmpl{"Maryna"}

\end{itemize}

\errheader
\begin{itemize}
    \item Error \ecode{5001} is returned if the \textit{assignmentid} is not specified.
    \item Error \ecode{5002} is returned if the \textit{responsiblelogin} is not specified.
    \item Error \ecode{5101} is returned if the \textit{assignmentid} is not a number (expected to be numeric).
    \item Error \ecode{5102} is returned if the \textit{assignmentid} is not 1 or more (expected to be 1 or a larger value).
    \item Error \ecode{5103} is returned if the \textit{assignmentid} is not an integer (expected to be an integer).
\end{itemize}



\subsection{Archive an assignment}

\begin{lstlisting}[style=CommandLineStyle]
.assignments_archive assignmentid
\end{lstlisting}

\paramsheader
\begin{itemize}

    \item \option{assignmentid} - numeric identifier of the assignment, for
    example:
    \xmpl{1007}

\end{itemize}

\errheader
\begin{itemize}
    \item Error \ecode{5001} is returned if the \textit{assignmentid} is not specified.
    \item Error \ecode{5101} is returned if the \textit{assignmentid} is not a number (expected to be numeric).
    \item Error \ecode{5102} is returned if the \textit{assignmentid} is not 1 or more (expected to be 1 or a larger value).
    \item Error \ecode{5103} is returned if the \textit{assignmentid} is not an integer (expected to be an integer).
\end{itemize}



\subsection{Unarchive an assignment}

\begin{lstlisting}[style=CommandLineStyle, showlines=true]

.assignments_unarchive assignmentid

\end{lstlisting}

\paramsheader
\begin{itemize}

    \item \option{assignmentid} - numeric identifier of the assignment, for example:
    \xmpl{1007}

\end{itemize}

\errheader
\begin{itemize}
    \item Error \ecode{5001} is returned if the \textit{assignmentid} is not specified.
    \item Error \ecode{5101} is returned if the \textit{assignmentid} is not a number (expected to be numeric).
    \item Error \ecode{5102} is returned if the \textit{assignmentid} is not 1 or more (expected to be 1 or a larger value).
    \item Error \ecode{5103} is returned if the \textit{assignmentid} is not an integer (expected to be an integer).
\end{itemize}


\subsection{Close an assignment}

\begin{lstlisting}[style=CommandLineStyle, showlines=true]

.assignments_close assignmentid

\end{lstlisting}

\paramsheader
\begin{itemize}

    \item \option{assignmentid} - numeric identifier of the assignment, for
    example:
    \xmpl{1007}

\end{itemize}

\errheader
\begin{itemize}
    \item Error \ecode{5001} is returned if the \textit{assignmentid} is not specified.
    \item Error \ecode{5101} is returned if the \textit{assignmentid} is not a number (expected to be numeric).
    \item Error \ecode{5102} is returned if the \textit{assignmentid} is not 1 or more (expected to be 1 or a larger value).
    \item Error \ecode{5103} is returned if the \textit{assignmentid} is not an integer (expected to be an integer).
\end{itemize}



\subsection{Change quantity for an assignment}

\begin{lstlisting}[style=CommandLineStyle, showlines=true]

.assignments_changequantity assignmentid number

\end{lstlisting}

\paramsheader
\begin{itemize}

    \item \option{assignmentid} - numeric identifier of the assignment, for
    example:
    \xmpl{1007}

    \item \option{number} - new quantity for the assignment, for example:
    \xmpl{10}

\end{itemize}

\errheader
\begin{itemize}
    \item Error \ecode{5001} is returned if the \textit{assignmentid} is not specified.
    \item Error \ecode{5002} is returned if the \textit{number} is not specified.
    \item Error \ecode{5003} is returned if the \textit{number} is invalid (must be an integer, more than 0, 0, or -1).
    \item Error \ecode{5101} is returned if the \textit{assignmentid} is not a number (expected to be numeric).
    \item Error \ecode{5102} is returned if the \textit{assignmentid} is not 1 or more (expected to be 1 or a larger value).
    \item Error \ecode{5103} is returned if the \textit{assignmentid} is not an integer (expected to be an integer).
\end{itemize}


\subsection{Get quantity setting for an assignment}

\begin{lstlisting}[style=CommandLineStyle, showlines=true]

.assignments_getquantitysettings assignmentid

\end{lstlisting}

\paramsheader
\begin{itemize}

    \item \option{assignmentid} - numeric identifier of the assignment, for
    example:
    \xmpl{1007}

    \item \option{number} - new quantity for the assignment, for example:
    \xmpl{10}

\end{itemize}

\savedres
\begin{compactitem}
    \item \savedresult{r(can\_change\_quantity)} - numeric 1=quantity can be changed or 0=quantity cannot be changed.
    \item \savedresult{r(status\_code)}  - numeric status code for API query (successful completion is indicated by code 200).
\end{compactitem}

\errheader
\begin{itemize}
    \item Error \ecode{5001} is returned if the \textit{assignmentid} is not specified.
    \item Error \ecode{5101} is returned if the \textit{assignmentid} is not a number (expected to be numeric).
    \item Error \ecode{5102} is returned if the \textit{assignmentid} is not 1 or more (expected to be 1 or a larger value).
    \item Error \ecode{5103} is returned if the \textit{assignmentid} is not an integer (expected to be an integer).
\end{itemize}


\subsection{Get audio audit status for an assignment}

\begin{lstlisting}[style=CommandLineStyle, showlines=true]

.assignments_getaudio assignmentid

\end{lstlisting}

\paramsheader
\begin{itemize}
    \item \option{assignmentid} - numeric identifier of the assignment, for example: \textit{1007}.
\end{itemize}

\savedres
\begin{compactitem}
    \item \savedresult{r(record\_audio)} - numeric 1=ON or 0=OFF.
    \item \savedresult{r(status\_code)}  - numeric status code for API query. Successful completion is indicated by code 200.
\end{compactitem}

\errheader
\begin{itemize}
    \item Error \ecode{5001} is returned if the \textit{assignmentid} is not specified.
    \item Error \ecode{5101} is returned if the \textit{assignmentid} is not a number (expected to be numberic).
    \item Error \ecode{5102} is returned if the \textit{assignmentid} is not 1 or more (expected to be 1 or a larger value).
    \item Error \ecode{5103} is returned if the \textit{assignmentid} is not an integer (expected to be an integer).
\end{itemize}

\subsection{Set audio audit status for an assignment}

\begin{lstlisting}[style=CommandLineStyle, showlines=true]

.assignments_setaudio assignmentid value

\end{lstlisting}

\paramsheader
\begin{itemize}

    \item \option{assignmentid} - numeric identifier of the assignment, for
    example:
    \xmpl{1007}

    \item \option{value} - numeric value for the audio audit status of the
    assignment (0=OFF, any other value=ON), for example:
    \xmpl{1}

\end{itemize}

\savedres
\begin{compactitem}
    \item \savedresult{r(record\_audio)} - numeric 1=ON or 0=OFF.
    \item \savedresult{r(status\_code)}  - numeric status code for API query
    (successful completion is indicated by code 200).
\end{compactitem}

\errheader
\begin{itemize}
    \item Error \ecode{5001} is returned if the \textit{assignmentid} is not specified.
    \item Error \ecode{5002} is returned if the \textit{value} of audio audit is not specified.
    \item Error \ecode{5101} is returned if the \textit{assignmentid} is not a number (expected to be numeric).
    \item Error \ecode{5102} is returned if the \textit{assignmentid} is not 1 or more (expected to be 1 or a larger value).
    \item Error \ecode{5103} is returned if the \textit{assignmentid} is not an integer (expected to be an integer).
\end{itemize}

\subsection{Get assignment's history}

\begin{lstlisting}[style=CommandLineStyle, showlines=true]

.assignments_history assignmentid frame

\end{lstlisting}

\paramsheader
\begin{itemize}
    \item \option{assignmentid} - numeric identifier of the assignment, for
    example:
    \xmpl{1007}

    \item \option{framename} - (optional) frame name where to save the
    assignment history, for example:
    \xmpl{"history"}\newline
    If not specified, the history is saved to the current frame.

\end{itemize}

\structheader{Assignment history}

\begin{compactitem}
    \datafield{str32 utc\_date } - timestamp when the action was taken, for
    example:
    \xmpl{"2020-10-05T15:33:30.317233Z"}

    \datafield{str32 action } - action type, for example:
    \xmpl{"Created"}

    \datafield{str32 actor\_name } - login of the actor that took the action,
    for example:
    \xmpl{"ValeriaHQ"}

    \datafield{strL additional\_data } - additional data that may or may not be
    present depending on the type of the action, for example:
    \xmpl{"\{'Comment': None, 'Responsible': 'SergiyInt'\}"}

\end{compactitem}

\errheader
\begin{itemize}
    \item Error \ecode{5001} is returned if the \textit{assignmentid} is not
    specified.
    \item Error \ecode{5101} is returned if the \textit{assignmentid} is not a
    number (expected to be numberic).
    \item Error \ecode{5102} is returned if the \textit{assignmentid} is not 1
    or more (expected to be 1 or a larger value).
    \item Error \ecode{5103} is returned if the \textit{assignmentid} is not an
    integer (expected to be an integer).
    \item Error \ecode{-10404} is returned if the assignment with
    \textit{assignmentid} is not found on the server.
\end{itemize}


\subsection{Create an assignment}

\begin{lstlisting}[style=CommandLineStyle, showlines=true]

.assignments_create ,
  responsible() qxguid()
  [qxversion() quantity()]
  [email() password() webmode()]
  [audio() comment() protected() data()]

\end{lstlisting}

\optsheader
\begin{itemize}

\item \option{responsible} - account name of a person designated responsible
for the assignment, for example: \newline
\textit{SergiyInt}\newline
NB: for web interviews the responsible must be a user in the interviewer role.

\item \option{qxguid()} - GUID of the questionnaire.

\item \option{qxversion()} - (optional) numeric version of the questionnaire
      (as assigned when the questionnaire is imported to the HQ), for example:
      \xmpl{2} \newline
      When the option is not specified version 1 is implied.

\item \option{quantity()} - (optional) quantity of interviews requested in
      this assignment, for example:
      \xmpl{12}\newline
      When the option is not specified, quantity 1 is implied.

\item \option{email()} - (optional) respondent's email for sending the
      invitations and reminders for a web interview.

\item \option{password()} - (optional) password for accessing the web interview
      online.

\item \option{webmode()} - (optional) whether the assignment should be
      created in web mode. Specify 1 for web mode assignments, and 0 for regular
      assignments. If the option is not specified, regular assignments will be
      created.

\item \option{audio()} - (optional) whether audio audit should be collected for
      this assignment (1=on, 0=off). If the option is not specified, the audio
      audit will not be collected.

\item \option{comment()} - (optional) a comment to be recorded as an additional
      instruction regarding this assignment to the person who is designated
      responsible.

\item \option{protected()} - (optional) a space-delimited list of variables to
      be protected from modification by the interviewer (see the corresponding
      article in the documentation for Survey Solutions). For example,
      \textit{"memberscount plotslist"}. If the option is not specified, no
      variables are protected.

\item \option{data()} - (optional) output of the \textit{\textbf{.buildvars}}
      method or manually constructed preloading data in the same format, for
      example:
\begin{lstlisting}[style=CommandLineStyle, showlines=true]

`"{"Variable":"age", "Answer":41}"'

\end{lstlisting}

\end{itemize}


NB: Remember to use Stata's compound double quotes to envelop the content in
the \option{data()} option since it contains quotes of its own:

\begin{lstlisting}[style=CommandLineStyle, showlines=true]

local d = ///
  `"{"Variable":"region", "' + ///
  `""Answer":5}, "' + ///
  `"{"Variable":"address", "' + ///
  `""Answer":"1234 Test St., My City, USA"}"'

.s.assignments_create , ... data (`"`d'"')

\end{lstlisting}


\savedres
\begin{compactitem}

    \item \savedresult{r(status\_code)} - numeric status code for the HTTP
    request corresponding to this query, for example:
    \xmpl{201}\newline
    (Successful completion is indicated by code 200 or 201)

    \item \savedresult{r(Id)} - numeric assignment ID, for example:
    \xmpl{1226612}

    \item \savedresult{r(InterviewsCount)} - number of interviews existing
    originating from this assignment, (expected always 0) for example:
    \xmpl{0}

    \item \savedresult{r(Archived)} - numeric presentation of archival status
    (1=archived, 0=not archived), (expected always 0, not archived), for
    example:
    \xmpl{0}

    \item \savedresult{r(WebMode)} - numeric presentation of the mode
    (1=Web mode, 0=Not web mode) for this assignment, for example:
    \xmpl{1}

    \item \savedresult{r(IsAudioRecordingEnabled)} - flag whether audio
    recording is enabled for this assignment (1=ON or 0=OFF), for example:
    \xmpl{0}

    \item \savedresult{r(Email)} - email of the web interview respondent, for
    example:
    \xmpl{"test@site.org"}

    \item \savedresult{r(CreatedAtUtc)} - timestamp of the creation of the
    assignment (in UTC), for example:
    \xmpl{"2022-10-18T22:01:11.351222Z"}

    \item \savedresult{r(UpdatedAtUtc)} - timestamp of the last update of the
    assignment (in UTC), for example:
    \xmpl{"2022-10-18T22:01:11.351222Z"}\newline
    (expected to be equal to timestamp of creation for newly created assignments)

    \item \savedresult{r(QuestionnaireId)} - identity of the questionnaire, for
    example:
    \xmpl{"8352f83fe9eb4e48888142402a49ce3e\$1"}

    \item \savedresult{r(ResponsibleName)} - login of the responsible, for
    example:
    \xmpl{"SergiyInt"}

    \item \savedresult{r(ResponsibleId)} - GUID of the responsible, for example:
    \xmpl{"7607263d-ad66-4cc6-a175-42d3e6a57745"}

    \item \savedresult{r(WebInterviewLink)} - web link to start an interview
    from this assignment, for example:
    \xmpl{"https://demo.mysurvey.solutions/census/WebInterview/V9FNU9PZ/Start"}

\end{compactitem}

\errheader
\begin{itemize}
    \item Error \ecode{7001} is returned if the \textit{password} option is
    specified, but the assignment is created not in web mode (option webmode is
    0).

    \item Error \ecode{7002} is returned if the \textit{email} option is
    specified, but the assignment is created not in web mode (option webmode is
    0).

    \item Error \ecode{7003} is returned if the \textit{quantity} option is
    more than 1, but the assignment is created not in web mode (option webmode is
    0).

    \item Error \ecode{7004} is returned if the \textit{quantity} option is
    more than 10,000, which is a Survey Solutions limit for finite assignments.
\end{itemize}
