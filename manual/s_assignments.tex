\section{Assignments}
\putimage{pexels-pixabay-164686.jpg}

\subsection{Get details of an assignment}

\begin{lstlisting}[style=CommandLineStyle]
.assignments_getdetails assignmentid
\end{lstlisting}

\paramsheader
\begin{itemize}
    \item \option{assignmentid} - numeric identifier of the assignment, for
    example: \textit{1007}.
\end{itemize}

\savedres
\begin{compactitem}
    \item r(CreatedAtUtc) - timestamp (date and time) of the creation of this
    assignment (in UTC), for example: \textit{"2021-05-17T14:41:46.824627Z"}.
    \item r(UpdatedAtUtc) - timestamp (date and time) of the last update (in
    UTC), for example: \textit{"2021-05-27T20:45:39.778723Z"}.
    \item r(QuestionnaireId)  - identity of the questionnaire on which this
    assignment is based. Identity is the questionnaire GUID and version
    separated with the \$ sign, for example: \newline
    \textit{"0434573c67b34d93b1f0799fb042f9e6\$1"}.
    \item r(ResponsibleName) - login name of the current responsible user,
    for example: \textit{"Natalia"}.
    \item r(ResponsibleId) - GUID of the current responsible user, for example:
    \textit{"71dc5a0f-6557-4945-9b86-0197699ee475"}.
\end{compactitem}

\vskip16pt
Additionally, for each identifying question, \#=1,2,...:
\begin{compactitem}
    \item r(id\_guid\#) - internally assigned GUID of the question,
    for example: \newline \textit{"c03979a9157304fcfa9bc1d3988433d7"}.
    \item r(id\_variable\#) - variable name for the question,
    for example: \textit{"province"}.
    \item r(id\_answer\#) - recorded answer to the identifying question,
    for example: \textit{"Northern province"}.
\end{compactitem}

\errheader
\begin{itemize}
    \item Error \ecode{5001} is returned if the \textit{assignmentid} is not specified.
    \item Error \ecode{5101} is returned if the \textit{assignmentid} is not a number (expected to be numberic).
    \item Error \ecode{5102} is returned if the \textit{assignmentid} is not 1 or more (expected to be 1 or a larger value).
    \item Error \ecode{5103} is returned if the \textit{assignmentid} is not an integer (expected to be an integer).
    \item Error \ecode{-10404} is returned if an assignment with \textit{assignmentid} is not found on the server.
\end{itemize}


\subsection{Assign an assignment to a new responsible}

\begin{lstlisting}[style=CommandLineStyle]
.assignments_assign assignmentid responsiblelogin
\end{lstlisting}

\paramsheader
\begin{itemize}
    \item \option{assignmentid} - numeric identifier of the assignment, for example: \textit{1007}.
    \item \option{responsiblelogin} - account name of the new responsible, for example: \textit{"Maryna"}.
\end{itemize}

\errheader
\begin{itemize}
    \item Error \ecode{5001} is returned if the \textit{assignmentid} is not specified.
    \item Error \ecode{5002} is returned if the \textit{responsiblelogin} is not specified.
    \item Error \ecode{5101} is returned if the \textit{assignmentid} is not a number (expected to be numberic).
    \item Error \ecode{5102} is returned if the \textit{assignmentid} is not 1 or more (expected to be 1 or a larger value).
    \item Error \ecode{5103} is returned if the \textit{assignmentid} is not an integer (expected to be an integer).
\end{itemize}



\subsection{Archive an assignment}

\begin{lstlisting}[style=CommandLineStyle]
.assignments_archive assignmentid
\end{lstlisting}

\paramsheader
\begin{itemize}
    \item \option{assignmentid} - numeric identifier of the assignment, for example: \textit{1007}.
\end{itemize}

\errheader
\begin{itemize}
    \item Error \ecode{5001} is returned if the \textit{assignmentid} is not specified.
    \item Error \ecode{5101} is returned if the \textit{assignmentid} is not a number (expected to be numberic).
    \item Error \ecode{5102} is returned if the \textit{assignmentid} is not 1 or more (expected to be 1 or a larger value).
    \item Error \ecode{5103} is returned if the \textit{assignmentid} is not an integer (expected to be an integer).
\end{itemize}



\subsection{Unarchive an assignment}

\begin{lstlisting}[style=CommandLineStyle]
.assignments_unarchive assignmentid
\end{lstlisting}

\paramsheader
\begin{itemize}
    \item \option{assignmentid} - numeric identifier of the assignment, for example: \textit{1007}.
\end{itemize}

\errheader
\begin{itemize}
    \item Error \ecode{5001} is returned if the \textit{assignmentid} is not specified.
    \item Error \ecode{5101} is returned if the \textit{assignmentid} is not a number (expected to be numberic).
    \item Error \ecode{5102} is returned if the \textit{assignmentid} is not 1 or more (expected to be 1 or a larger value).
    \item Error \ecode{5103} is returned if the \textit{assignmentid} is not an integer (expected to be an integer).
\end{itemize}


\subsection{Close an assignment}

\begin{lstlisting}[style=CommandLineStyle]
.assignments_close assignmentid
\end{lstlisting}

\paramsheader
\begin{itemize}
    \item \option{assignmentid} - numeric identifier of the assignment, for example: \textit{1007}.
\end{itemize}

\errheader
\begin{itemize}
    \item Error \ecode{5001} is returned if the \textit{assignmentid} is not specified.
    \item Error \ecode{5101} is returned if the \textit{assignmentid} is not a number (expected to be numberic).
    \item Error \ecode{5102} is returned if the \textit{assignmentid} is not 1 or more (expected to be 1 or a larger value).
    \item Error \ecode{5103} is returned if the \textit{assignmentid} is not an integer (expected to be an integer).
\end{itemize}



\subsection{Change quantity for an assignment}

\begin{lstlisting}[style=CommandLineStyle]
.assignments_changequantity assignmentid number
\end{lstlisting}

\paramsheader
\begin{itemize}
    \item \option{assignmentid} - numeric identifier of the assignment, for example: \textit{1007}.
    \item \option{number} - new quantity for the assignment, for example: \textit{10}.
\end{itemize}

\errheader
\begin{itemize}
    \item Error \ecode{5001} is returned if the \textit{assignmentid} is not specified.
    \item Error \ecode{5002} is returned if the \textit{number} is not specified.
    \item Error \ecode{5003} is returned if the \textit{number} is invalid (must be an integer, more than 0, 0, or -1).
    \item Error \ecode{5101} is returned if the \textit{assignmentid} is not a number (expected to be numberic).
    \item Error \ecode{5102} is returned if the \textit{assignmentid} is not 1 or more (expected to be 1 or a larger value).
    \item Error \ecode{5103} is returned if the \textit{assignmentid} is not an integer (expected to be an integer).
\end{itemize}


\subsection{Get quantity setting for an assignment}

\begin{lstlisting}[style=CommandLineStyle]
.assignments_getquantitysettings assignmentid
\end{lstlisting}

\paramsheader
\begin{itemize}
    \item \option{assignmentid} - numeric identifier of the assignment, for example: \textit{1007}.
    \item \option{number} - new quantity for the assignment, for example: \textit{10}.
\end{itemize}

\savedres
\begin{compactitem}
    \item r(can\_change\_quantity) - numeric 1=quantity can be changed or 0=quantity cannot be changed.
    \item r(status\_code)  - numeric status code for API query. Successful completion is indicated by code 200.
\end{compactitem}

\errheader
\begin{itemize}
    \item Error \ecode{5001} is returned if the \textit{assignmentid} is not specified.
    \item Error \ecode{5101} is returned if the \textit{assignmentid} is not a number (expected to be numberic).
    \item Error \ecode{5102} is returned if the \textit{assignmentid} is not 1 or more (expected to be 1 or a larger value).
    \item Error \ecode{5103} is returned if the \textit{assignmentid} is not an integer (expected to be an integer).
\end{itemize}


\subsection{Get audio audit status for an assignment}

\begin{lstlisting}[style=CommandLineStyle]
.assignments_getaudio assignmentid
\end{lstlisting}

\paramsheader
\begin{itemize}
    \item \option{assignmentid} - numeric identifier of the assignment, for example: \textit{1007}.
\end{itemize}

\savedres
\begin{compactitem}
    \item r(record\_audio) - numeric 1=ON or 0=OFF.
    \item r(status\_code)  - numeric status code for API query. Successful completion is indicated by code 200.
\end{compactitem}

\errheader
\begin{itemize}
    \item Error \ecode{5001} is returned if the \textit{assignmentid} is not specified.
    \item Error \ecode{5101} is returned if the \textit{assignmentid} is not a number (expected to be numberic).
    \item Error \ecode{5102} is returned if the \textit{assignmentid} is not 1 or more (expected to be 1 or a larger value).
    \item Error \ecode{5103} is returned if the \textit{assignmentid} is not an integer (expected to be an integer).
\end{itemize}

\subsection{Set audio audit status for an assignment}

\begin{lstlisting}[style=CommandLineStyle]
.assignments_setaudio assignmentid value
\end{lstlisting}

\paramsheader
\begin{itemize}
    \item \option{assignmentid} - numeric identifier of the assignment, for example: \textit{1007}.
    \item \option{value} - numeric value for the audio audit status of the assignment (0=OFF, any other value=ON), for example: \textit{1}.

\end{itemize}

\savedres
\begin{compactitem}
    \item r(record\_audio) - numeric 1=ON or 0=OFF.
    \item r(status\_code)  - numeric status code for API query. Successful completion is indicated by code 200.
\end{compactitem}

\errheader
\begin{itemize}
    \item Error \ecode{5001} is returned if the \textit{assignmentid} is not specified.
    \item Error \ecode{5002} is returned if the \textit{value} of audio audit is not specified.
    \item Error \ecode{5101} is returned if the \textit{assignmentid} is not a number (expected to be numberic).
    \item Error \ecode{5102} is returned if the \textit{assignmentid} is not 1 or more (expected to be 1 or a larger value).
    \item Error \ecode{5103} is returned if the \textit{assignmentid} is not an integer (expected to be an integer).
\end{itemize}

\subsection{Get assignment's history}

\begin{lstlisting}[style=CommandLineStyle]
.assignments_history assignmentid frame
\end{lstlisting}

\paramsheader
\begin{itemize}
    \item \option{assignmentid} - numeric identifier of the assignment, for example: \textit{1007}.
    \item \option{framename} - (optional) frame name where to save the assignment history, for example: \textit{"history"}. If not specified, the history is saved to the current frame.

\end{itemize}

\structheader{Assignment history}

\begin{compactitem}
    \datafield{str32 utc\_date } (for example, "\textit{2020-10-05T15:33:30.317233Z}")
    \datafield{str32 action } (for example, "\textit{Created}")
    \datafield{str32 actor\_name } (for example, "\textit{ValeriaHQ}")
    \datafield{strL additional\_data } (for example, "\textit{\{'Comment': None, 'Responsible': 'SergiyInt'\}}")
\end{compactitem}

\errheader
\begin{itemize}
    \item Error \ecode{5001} is returned if the \textit{assignmentid} is not specified.
    \item Error \ecode{5101} is returned if the \textit{assignmentid} is not a number (expected to be numberic).
    \item Error \ecode{5102} is returned if the \textit{assignmentid} is not 1 or more (expected to be 1 or a larger value).
    \item Error \ecode{5103} is returned if the \textit{assignmentid} is not an integer (expected to be an integer).    
    \item Error \ecode{-10404} is returned if the assignment with \textit{assignmentid} is not found on the server.
\end{itemize}


\subsection{Create an assignment}

\begin{lstlisting}[style=CommandLineStyle]
.assignments_create , responsible() qxguid() [qxversion() quantity()]
             [email() password() webmode()]
             [audio() comment() protected() data()]
\end{lstlisting}

\optsheader
\begin{itemize}

\item \option{responsible} - account name of a person designated responsible
for the assignment, for example: \textit{SergiyInt}. NB: for web interviews the
responsible must be a user in the interviewer role.

\item \option{qxguid()} - GUID of the questionnaire.

\item \option{qxversion()} - (optional) numeric version of the questionnaire
      (as assigned when the questionnaire is imported to the HQ), for example:
      \textit{2}. When the option is not specified version 1 is implied.

\item \option{quantity()} - (optional) quantity of interviews requested in
      this assignment, for example: \textit{12}. When the option is not
      specified, quantity 1 is implied.

\item \option{email()} - (optional) respondent's email for sending the
      invitations and reminders for a web interview.

\item \option{password()} - (optional) password for accessing the web interview
      online.

\item \option{webmode()} - (optional) whether the assignment should be
      created in web mode. Specify 1 for web mode assignments, and 0 for regular
      assignments. If the option is not specified, regular assignments will be
      created.

\item \option{audio()} - (optional) whether audio audit should be collected for
      this assignment (1=on, 0=off). If the option is not specified, the audio
      audit will not be collected.

\item \option{comment()} - (optional) a comment to be recorded as an additional
      instruction regarding this assignment to the person who is designated
      responsible.

\item \option{protected()} - (optional) a space-delimited list of variables to
      be protected from modification by the interviewer (see the corresponding
      article in the documentation for Survey Solutions). For example,
      \textit{"memberscount plotslist"}. If the option is not specified, no
      variables are protected.

\item \option{data()} - (optional) output of the \textit{\textbf{.buildvars}}
      method or manually constructed preloading data in the same format, for
      example: \textit{"{"Variable":"age", "Answer":41}"}

\end{itemize}

\errheader
\begin{itemize}
    \item Error \ecode{7001} is returned if the \textit{password} option is
    specified, but the assignment is created not in web mode (option webmode is
    0).

    \item Error \ecode{7002} is returned if the \textit{email} option is
    specified, but the assignment is created not in web mode (option webmode is
    0).

    \item Error \ecode{7003} is returned if the \textit{quantity} option is
    more than 1, but the assignment is created not in web mode (option webmode is
    0).
\end{itemize}
