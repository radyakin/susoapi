\section{Workspaces}
\putimage{pexels-tima-miroshnichenko-5453837.jpg}


Workspaces management is a privilege and responsibility. Only some of the
methods listed below are available to the API users. Other methods denoted
with an asterisk (*) in the description are available only to the Survey
Solutions administrator. You can switch the account to be used for API
queries already after the initialization of the API client, if necessary:

\vskip16pt

\begin{lstlisting}[style=CommandLineStyle, showlines=true]

.s.username="admin"
.s.password="Some123Secret"

\end{lstlisting}

Note also, that the administrator always has access to all the workspaces on
the server.


\subsection{Obtain list of workspaces}

\begin{lstlisting}[style=CommandLineStyle, showlines=true]

.workspaces_list ,
    [frame()] [userguid()] [includedisabled]

\end{lstlisting}

\optsheader
\begin{itemize}

  \item \option{frame()} - (optional) name of the frame where the list of the
        workspaces must be saved, for example:
        \xmpl{"wspaces"}

  \item \option{userguid()} - (optional) GUID of the user, if specified the
        list will include only the workspaces, to which the specified user is
        granted access to.

  \item \option{includedisabled} - (optional), if specified the list will also
        include disabled workspaces; if not specified, then only enabled
        workspaces.

\end{itemize}

\textbf{Workspaces list frame data structure}

\begin{compactitem}

    \datafield{string name }  - workspace name, for example:
    \xmpl{"census"}

    \datafield{string display\_name } - workspace display name, for example:
    \xmpl{"Population Census 2021"}

    \datafield{string disabled\_at\_utc } - date and time when the workspace
    was disabled, in UTC, for example:
    \xmpl{"2020-10-05T15:33:30.317233Z"}

\end{compactitem}

*NB: The exact storage types of the string variables depend on the content
     length and may vary.


\subsection{Get details of a workspace}

\begin{lstlisting}[style=CommandLineStyle, showlines=true]

.workspaces_getdetails name

\end{lstlisting}

\paramsheader
\begin{itemize}

    \item \option{name} - name of the workspace, for example:
    \xmpl{lfs}

\end{itemize}

\errheader
\begin{itemize}
    \item Error \ecode{5001} is returned if the workspace \textit{name} is
    not specified.
\end{itemize}

\savedres
\begin{compactitem}

    \item \savedresult{r(Name)} - specified name of the workspace.

    \item \savedresult{r(DisplayName)} - display name that is attributed to
    this workspace.

    \item \savedresult{r(DisabledAtUtc)} - blank for workspaces that are not
    disabled, for disabled workspaces timestamp indicating when the workspace
    was disabled.

\end{compactitem}


\subsection{*Create a workspace}

\begin{lstlisting}[style=CommandLineStyle, showlines=true]

.workspaces_update name [title]

\end{lstlisting}

\paramsheader
\begin{itemize}

  \item \option{name} - name of the new workspace, for example:
  \xmpl{lfs}

  \item \option{title} - (optional) the title of the new workspace, for example:
        \xmpl{"Labour force survey 2021q4"}\newline
        The name of the workspace will be used as the title if no title is
        specified.

\end{itemize}

\errheader
\begin{itemize}
   \item Error \ecode{5001} is returned if the workspace \textit{name} is
         not specified.
\end{itemize}


\subsection{*Delete a workspace}

\begin{lstlisting}[style=CommandLineStyle, showlines=true]

.workspaces_delete name , irreversible

\end{lstlisting}

\paramsheader
\begin{itemize}

  \item \option{name} - name of the workspace to be deleted, for example:
  \xmpl{lfs}

  \item \option{irreversible} - literally the word \textit{irreversible}. This
        option is not optional! By issuing this option the user confirms that
        the command may cause irreversible loss of data.

\end{itemize}

\errheader
\begin{itemize}

   \item Error \ecode{5001} is returned if the workspace \textit{name} is
         not specified.

   \item Error \ecode{198} is returned if the option \textit{irreversible} is
         not specified.

\end{itemize}


\subsection{*Obtain workspace status}

\begin{lstlisting}[style=CommandLineStyle, showlines=true]

.workspaces_status name

\end{lstlisting}

\paramsheader
\begin{itemize}

  \item \option{name} - name of the workspace the status of which is to be
        examined, for example:
  \xmpl{lfs}

\end{itemize}

\errheader
\begin{itemize}
   \item Error \ecode{5001} is returned if the workspace \textit{name} is
         not specified.
\end{itemize}

\savedres
\begin{compactitem}

    \item \savedresult{r(WorkspaceName)} - specified name of the workspace,
          for example:
          \xmpl{"census"}

    \item \savedresult{r(WorkspaceDisplayName)} - display name that is
          attributed to this workspace, for example:
          \xmpl{"Population Census 2021"}

    \item \savedresult{r(CanBeDeleted)} - a flag indicating whether the
          workspace may be deleted (returned value is 1) or not (returned value
          is 0).

    \item \savedresult{r(ExistingQuestionnairesCount)} - number of
          questionnaires that are imported to this workspace.

    \item \savedresult{r(InterviewersCount)} - number of interviewer accounts
          in this workspace.

    \item \savedresult{r(SupervisorsCount)} - number of supervisor accounts in
          this workspace.

    \item \savedresult{r(MapsCount)} - number of maps in this workspace.

\end{compactitem}


\subsection{*Disable a workspace}

\begin{lstlisting}[style=CommandLineStyle, showlines=true]

.workspaces_disable name

\end{lstlisting}

\paramsheader
\begin{itemize}

    \item \option{name} - name of the workspace, for example:
    \xmpl{lfs}

\end{itemize}

\errheader
\begin{itemize}
    \item Error \ecode{5001} is returned if the workspace \textit{name} is
    not specified.
\end{itemize}

\subsection{*Enable a workspace}

\begin{lstlisting}[style=CommandLineStyle, showlines=true]

.workspaces_enable name

\end{lstlisting}

\paramsheader
\begin{itemize}

    \item \option{name} - name of the workspace, for example:
    \xmpl{lfs}

\end{itemize}

\errheader
\begin{itemize}
    \item Error \ecode{5001} is returned if the workspace \textit{name} is
    not specified.
\end{itemize}


\subsection{*Assign users to workspaces}

\begin{lstlisting}[style=CommandLineStyle, showlines=true]

.workspaces_assign mode, users() workspaces()

\end{lstlisting}

\vskip10pt
The \textbf{mode} can be one of the following:
\vskip10pt
\begin{compactitem}
  \item \textbf{add} - give the user(s) access to specified workspace(s)
                  in addition to any other workspaces she currently
                  has access to.
  \item \textbf{remove} - take away the access permission to the specified
                  workspace(s) leaving any other access permissions
                  unaffected.
  \item \textbf{assign} - replace current access permissions with the
                  access only to specified workspaces.
\end{compactitem}

\optsheader

\begin{itemize}

  \item \option{users()} - one or more user account names, for example:
        \xmpl{"SergiyInt"} \newline
        Multiple account names must be delimited with spaces, like so:
        \xmpl{"SergiyInt SandraSup"}

  \item \option{workspaces()} - one or more names of the workspaces, for
        example:
        \xmpl{"agcensus"} \newline
        Multiple workspace names must be delimited with spaces, like so: \newline
        \xmpl{"census lfs pricesvy"}

\end{itemize}

\subsection{Update a workspace title}

\begin{lstlisting}[style=CommandLineStyle]
.workspaces_update name title
\end{lstlisting}

\paramsheader
\begin{itemize}

  \item \option{name} - name of the workspace, for example:
  \xmpl{lfs}

  \item \option{title} - new title of the workspace, for example:
  \xmpl{"Labour force survey 2021q4"}

\end{itemize}

\errheader
\begin{itemize}
  \item Error \ecode{5001} is returned if the workspace \textit{name} is not specified.
  \item Error \ecode{5002} is returned if the workspace \textit{title} is not specified.
\end{itemize}
