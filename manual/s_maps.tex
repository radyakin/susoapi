\section{Maps}

\putimage{pexels-andrew-neel-2859169.jpg}

\subsection{Upload a map to the server.}
\begin{lstlisting}[style=CommandLineStyle, showlines=true]

.maps_uploadmaps ,
    mapsfile(string)
    frame(string)

\end{lstlisting}

Call this API endpoint to upload a map to the server. The map must then be
assigned to one or more interviewers by calling the \textbf{.maps\_addusertomap}
endpoint. The zip archive containing the maps may include one or multiple maps,
including maps of different types (such as maps in .tpk, .mmpk, .tiff, etc formats).

NB: use the forward slash '\slash' rather than backward slash '\textbackslash' for
separating the folders in the path to file in the \textit{mapsfile} parameter.

\paramsheader
\begin{itemize}

  \item \option{mapsfile} - name of the zip archive containing one or multiple
        map files to be uploaded to the server, for example, \xmpl{\textquotedbl
        newmaps.zip\textquotedbl}, if the file is not in the current
        directory, include the full path to the file, for example: \newline
        \xmpl{\textquotedbl C:/data/gis/maps/newmaps.zip \textquotedbl}

  \item \option{frame} - name of a new frame that will be created to contain
        information about the individual map files that were uploaded, for
        example: \newline
        \xmpl{\textquotedbl \textit{loadedmaps}\textquotedbl}

\end{itemize}

\errheader
\begin{itemize}
    \item Error \ecode{5001} is returned if the server does not support
    maps uploading.
\end{itemize}

\savedres
\begin{compactitem}
    \item \savedresult{r(nMapsUploaded)} - number of successfully uploaded maps.
\end{compactitem}



\subsection{Get list of maps and maps-to-user assignments}
\begin{lstlisting}[style=CommandLineStyle, showlines=true]

.maps_getmaps maps [users]

\end{lstlisting}

\paramsheader
\begin{itemize}
  \item \option{maps} - frame name for placing the list of maps currently on the server
  \item \option{users} - (optional) frame name for placing the list of users-to-maps assignments.
\end{itemize}

\structheader{Maps}
\begin{compactitem}
  \datafield{string* filename }
  \datafield{string* importdateutc }
  \datafield{long size }
  \datafield{long users }
  \datafield{long wkid }
  \datafield{double minscale }
  \datafield{double maxscale }
  \datafield{double xminval }
  \datafield{double xmaxval }
  \datafield{double yminval }
  \datafield{double ymaxval }
\end{compactitem}

*NB: The exact storage types of the string variables depend on the content
     length and may vary.

\structheader{User-maps}
\begin{compactitem}
  \datafield{string* filename }
  \datafield{string* username }
\end{compactitem}

*NB: The exact storage types of the string variables depend on the content
     length and may vary.


\subsection{Add an assignment of a map to a user.}
\begin{lstlisting}[style=CommandLineStyle, showlines=true]

.maps_addusertomap map user

\end{lstlisting}

\paramsheader
\begin{itemize}

  \item \option{map} - name of the map from the list of maps
        currently on the server, for example: \newline
        \xmpl{\textquotedbl ukraine.tif\textquotedbl}

  \item \option{user} - interviewer account name, for example: \newline
        \xmpl{\textquotedbl SergiyInt\textquotedbl}

\end{itemize}


\subsection{Delete an assignment of a map from a user.}
\begin{lstlisting}[style=CommandLineStyle, showlines=true]

.maps_deleteuserfrommap map user

\end{lstlisting}

\paramsheader
\begin{itemize}

  \item \option{map} - name of the map from the list of maps currently on
        the server, for example: \newline
        \xmpl{\textquotedbl \textit{ukraine.tif}\textquotedbl}

  \item \option{user} - interviewer account name, for example: \newline
         \xmpl{\textquotedbl SergiyInt\textquotedbl}

\end{itemize}


\subsection{Delete a map from the server.}
\begin{lstlisting}[style=CommandLineStyle, showlines=true]

.maps_deletemap map

\end{lstlisting}

\paramsheader
\begin{itemize}
  \item \option{map} - name of the map from the list of maps currently on
  the server, for example: \newline
  \xmpl{\textquotedbl \textit{ukraine.tif}\textquotedbl}
\end{itemize}
